\documentclass[11pt]{amsart}
\usepackage[margin=.5in]{geometry}                % See geometry.pdf to learn the layout options. There are lots.
\geometry{letterpaper}                   % ... or a4paper or a5paper or ... 
%\geometry{landscape}                % Activate for for rotated page geometry
%\usepackage[parfill]{parskip}    % Activate to begin paragraphs with an empty line rather than an indent
\usepackage{graphicx}
\usepackage{amssymb}
\usepackage{epstopdf}
\usepackage{paralist}
\DeclareGraphicsRule{.tif}{png}{.png}{`convert #1 `dirname #1`/`basename #1 .tif`.png}

\title{Duncan College Article: Room Draw}
\author{Duncan College}
%\date{}                                           % Activate to display a given date or no date

\begin{document}
\maketitle
%%%%%%%%%%%%%%%%%%%%%%%%%%%%%%%%%%%%%%%%%
\section{Order of Room Draw}
%%%%%%%%%%%%%%%%%%%%%%%%%%
\begin{enumerate}
\item Outline of Room Draw Order
	%%%%%%%%%%%%%
	\begin{enumerate}
	\item General
		%%%%%%%%%%%%%
		\begin{enumerate}
		\item Room draw will be organized in such a way that there is sufficient time for a group who fails to receive a room in a given round of draw to re-organize and enter the next round.
		\end{enumerate}
	\item The order of the draws is listed below
		%%%%%%%%%%%%%
		\begin{enumerate}
			\item President
			\item 5 Man Suites(5th B\&C)
			\item 6 Man Suites(2nd-5th A\&D)
			\item Open Suites(2nd-4th B\&C)
			\item Singles
			\item Doubles
		\end{enumerate}
	\end{enumerate}
%%%%%%%%%%%%%%%%%%%%%%%%%%%%%%%%%%%%%%%%%
\item Overview of the various draws
	%%%%%%%%%%%%
	\begin{enumerate}
		\item President's Draw
			%%%%%%%%%%%%%
			\begin{enumerate}
				\item The president has first pick of any of the draw categories listed above and may populate the remaining beds with anyone he/she chooses so long as those people were not jacked.
				\item The president does not have the power to bring people back who were jacked in room jack, or who chose to go off.
			\end{enumerate}
		\item 5 Man Suites(5th B\&C)
			%%%%%%%%%%%%%
			\begin{enumerate}
				\item The two 5 man suites on the 5th floor will be drawn second. This is done in orderto give groups that fail to draw either 5 man suite the opportunity to go in for 6
man suite draw.
			\end{enumerate}
		\item 6 Man Suites(2nd-5th A\&D)
			%%%%%%%%%%%%%
			\begin{enumerate}
				\item The eight 6 man suites will be drawn third.
			\end{enumerate}
		\item Open Suites(2nd-4th B\&C)
			%%%%%%%%%%%%%
			\begin{enumerate}
				\item Open suites may be drawn as groups of six, or eight people. Effectively giving each group the option to include the two singles outside the suite proper in their draw.
				\item There is no preference in draw order as a result of how many people are going for a suite.
				\item All open suites that are not drawn by groups will go into singles draw.
			\end{enumerate}
		\item Doubles
			%%%%%%%%%%%%%
			\begin{enumerate}
				\item The new O-Week Coordinators will communicate with the LVP and reserve doubles for freshman. The reserved freshman rooms must be distributed more or less evenly throughout all of the halls on floors 2 through 4. The LVP and O-Week Coordinators may choose to negotiate and change which rooms are reserved, but are under no obligation to do so. All non-reserved rooms are
drawn in doubles draw.
				\item In the event that more people want doubles than are available, an extra round of singles draw will be held after doubles draw.
			\end{enumerate}
	\end{enumerate}
\end{enumerate}
%%%%%%%%%%%%%%%%%%%%%%%%%%%%%%%%%%%%%%%%%
\section{Room Selection Order}
%%%%%%%%%%%%%%%%%%%%%%%%%%
\begin{enumerate}
\item General Outline
	%%%%%%%%%%%%%
	\begin{enumerate}
		\item The order in which people are given the chance to draw their rooms is based on point
system rooted in seniority. In this document when a class is referred to by it�s rising
status. Therefore, room draw is only concerned with sophomores, juniors, and seniors.
	\end{enumerate}
\item Point Values
	%%%%%%%%%%%%%
	\begin{enumerate}
		\item The point values are assigned as listed below.
			%%%%%%%%%%%%%
			\begin{enumerate}
				\item Seniors = 3
				\item Juniors = 2
				\item Sophomores = 1
				\item Any student who failed to receive their freshmen point = 0
			\end{enumerate}
		\item If someone is in their fourth year, but has not declared as a senior (Is planning on taking a 5th year and went through room jack as a junior) they will be classified as a Junior.
	\end{enumerate}
\item Ties and How they Function
	%%%%%%%%%%%%%
	\begin{enumerate}
		\item Every person brings a point value to a group; all of the point values of people going into
draw together are averaged for a single group value.
		\item All room draws will be done in four tiers.
			%%%%%%%%%%%%%
			\begin{enumerate}
				\item People/Groups with a value � 2.66 (Senior draw)
				\item People/Groups with a value � 1.66 (Junior draw)
				\item People/Groups with a value � 0.66 (Sophomore draw)
				\item People/Groups with a value � 0 (No Freshmen Point Draw)
			\end{enumerate}
		\item Once groups are sorted into tiers the groups in each tier are put into a randomized order to select rooms.
	\end{enumerate}

\end{enumerate}




\end{document}  