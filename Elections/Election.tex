\documentclass[11pt]{amsart}
\usepackage[margin=.5in]{geometry}                % See geometry.pdf to learn the layout options. There are lots.
\geometry{letterpaper}                   % ... or a4paper or a5paper or ... 
%\geometry{landscape}                % Activate for for rotated page geometry
%\usepackage[parfill]{parskip}    % Activate to begin paragraphs with an empty line rather than an indent
\usepackage{graphicx}
\usepackage{amssymb}
\usepackage{epstopdf}
\usepackage{paralist}
\DeclareGraphicsRule{.tif}{png}{.png}{`convert #1 `dirname #1`/`basename #1 .tif`.png}

\title{Duncan Election Code of Conduct}
\author{Duncan College}
%\date{}                                           % Activate to display a given date or no date

\begin{document}
\maketitle
%\section{}
%\subsection{}

\begin{enumerate}

\item Entering an Election
	\begin{enumerate}
	\item All interested and potential candidates must submit a declaration of intent (also referred
to as an application) to enter a specific race.
	\item Once a candidate has submitted their declaration of intent, they are considered �entered
in the race.�
	\item Once entered into the race, a candidate must adhere to all regulations listed within the
Duncan Election Code of Conduct.
	\end{enumerate}
	%%%%%%%%%%%%%%%%%%%%%%%%%%%%%%%%%%%%%%%%%%%%%%%
\item Candidate Regulations
	\begin{enumerate}
	\item Candidates may not post or advertise their candidacy on any electronic media or physical poster.
	\item Candidates may advertise their candidacy by word of mouth, but may not advertise to a captive audience.
	\item Candidates may not engage in any offensive conduct specifically intended to hurt another candidate, interpreted as, but not limited to, disparaging personal statements, offensive slader, and other direct personal and offensive assaults.
	\item Candidates may not solicit votes from any person under the pretense of future benefit or gift to that person.
	\item Any candidate not in full compliance with the Duncan Election Code of Conduct is subject to removal from the candidacy by vote of the Duncan Executive Committee or if elected, removal from office, under the terms of impeachment of elected offices. 
	\end{enumerate}
	%%%%%%%%%%%%%%%%%%%%%%%%%%%%%%%%%%%%%%%%%%%%%%%
\item Determination of Elections
	\begin{enumerate}
	\item All Election ballots shall be preferential.
	\item Preferential voting and the counting of ballots shall be conducted according to Robert�s
Rules of Order, 10th Edition.
	\item The winner of each race shall be the candidate who receives a majority vote. For races
where there is more than one candidate, this will be after the final elimination according
to preferential voting rules.
	\item The counting will be done by the Legislative Vice President and two tellers, who shall
be members of the college. The tellers will witness the ballot counting and check the
accuracy of the Legislative Vice President to ensure accurate results.
	\item The results of an election will be posted publicly within one day of being officially
determined.
	\item Any student may submit a petition of dissent or objection to the Legislative Vice
President up to one week from the results being announced objecting to the conduct of the elections, a candidate�s conduct, or other concerns that would compromise the legitimacy and integrity of the college and the elections.
	\end{enumerate}

\end{enumerate}


\end{document}  