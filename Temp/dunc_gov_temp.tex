\documentclass[11pt]{amsart}
\usepackage[margin=.5in]{geometry}                % See geometry.pdf to learn the layout options. There are lots.
\geometry{letterpaper}                   % ... or a4paper or a5paper or ... 
%\geometry{landscape}                % Activate for for rotated page geometry
%\usepackage[parfill]{parskip}    % Activate to begin paragraphs with an empty line rather than an indent
\usepackage{graphicx}
\usepackage{amssymb}
\usepackage{epstopdf}
\usepackage{paralist}
\DeclareGraphicsRule{.tif}{png}{.png}{`convert #1 `dirname #1`/`basename #1 .tif`.png}

\title{Duncan College Article: Impeachment and Resignation}
\author{Duncan College}
%\date{}                                           % Activate to display a given date or no date

\begin{document}
\maketitle
\begin{enumerate}
	\item Impeachable Officials
	\begin{enumerate}
		\item Any official within Duncan can have impeachment charges brought against it by any
member of the college if the official is:
		\begin{enumerate}
			\item Elected in a general election
			\item Appointed by a member of the Duncan Executive Committee
		\end{enumerate}
	\end{enumerate}
	\item Procedure for Elected Officials
	\begin{enumerate}
		\item A member of Duncan College must submit to the Legislative Vice President a formal
petition of complaint detailing an incident(s) or reasoning for impeachment of the elected official(s). If the charges would be brought against the Legislative Vice President, then the petition is instead submitted to the Duncan President.
		\item The Legislative Vice President (or President in the case the charges are brought against the Legislative Vice President) will review the petition to ensure it is in the proper form before informing the official(s) that the charges would be brought against, as well as transmitting copies of the petition to the members of the Duncan Executive Committee and the class representatives.
		\item Once petitions have been distributed, the Chief Justice will set a date for a trial with a jury comprised of the class representatives, with the Chief Justice presiding and non- voting to ensure proceedings are carried out justly. In the event that the charges are brought against the Chief Justice, the Duncan President shall preside over the trial.
		\item The court will hear evidence and testimony from the accuser(s) and will subsequently hear a defense from the impeached official. The court will then have a period of time
to investigate the matter through questions to the accuser, the impeached official, and any witnesses brought to the proceedings. After the court has concluded investigation, the class representatives will adjourn to a private meeting to discuss the proceedings and arrive at a decision. The decision of the class representatives shall require a two- thirds majority. The court will immediately re-convene once the justices have reached a conclusion to announce the majority opinion. The decision of the class representatives shall be effective immediately after the adjournment of the court.
	\end{enumerate}
	\item Procedure for Appointed Officials
	\begin{enumerate}
		\item A member of Duncan College must submit to the Legislative Vice President a formal
petition of complaint detailing an incident(s) or reasoning for impeachment of the elected
official(s).
		\item The Legislative Vice President will review the petition to ensure it is in the proper form
before informing the official(s) that the changes would be brought against, as well as
transmitting copies of the petition to the members of the Duncan Executive Committee.
		\item Once petitions have been distributed, the President will convene the Duncan Executive
Committee to deliberate over the impeached officials and vote upon their removal.
		\item The Duncan Executive Committee will hear evidence and testimony from the accuser(s)
and will subsequently hear a defense from the impeached official. The Duncan
?
Executive Committee will then have a period of time to investigate the matter through questions to the accuser, the impeached official, and any witnesses brought to the proceedings. After the Executive Committee has concluded their investigation, the Executive Committee will adjourn to a private meeting to discuss the proceedings and arrive at a decision. The decision of the Executive Committee shall require a two-thirds majority. The Executive Committee will inform the appointed officials of their decision, taking effect immediately thereafter.
	\end{enumerate}
	\item Requirements for Impeachment Charges
	\begin{enumerate}
		\item A petition for impeachment may be brought against an official for exhibiting
malfeasance, exhibiting extreme partiality, exceeding authority, or failing to meet vested responsibilities.
	\end{enumerate} 
	\item Filling Empty Offices
	\begin{enumerate}
		\item A vacant office must be filled at the earliest time possible.
		\item The same procedure is used to fill vacant offices during the year as would normally be
used.
		\item For offices that include more than one person, the remaining individuals shall establish
a nomination procedure, such as, but not limited to, petitions of interest or applications to the entire current Duncan population. Once applications have been received, the remaining individuals shall nominate to forum candidates to fill the vacancies. Forum must approve the nominations by a simple majority.
	\end{enumerate}
\end{enumerate}




\end{document}  