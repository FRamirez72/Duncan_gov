\documentclass[11pt]{amsart}
\usepackage[margin=.5in]{geometry}                % See geometry.pdf to learn the layout options. There are lots.
\geometry{letterpaper}                   % ... or a4paper or a5paper or ... 
%\geometry{landscape}                % Activate for for rotated page geometry
%\usepackage[parfill]{parskip}    % Activate to begin paragraphs with an empty line rather than an indent
\usepackage{graphicx}
\usepackage{amssymb}
\usepackage{epstopdf}
\usepackage{paralist}
\DeclareGraphicsRule{.tif}{png}{.png}{`convert #1 `dirname #1`/`basename #1 .tif`.png}

\title{Freshman Service Points}
\author{Duncan College}
%\date{}                                           % Activate to display a given date or no date

\begin{document}
\maketitle
%\section{}
%\subsection{}

\section{Freshmen Classification}
\begin{compactenum}
\item For the purposes of freshmen service points, freshmen shall be defined as 
students who matriculated Rice in the fall semester of that academic year without 
having completed a freshmen year at another university or college.
\item A student will also be given freshmen status if they matriculated Rice in the fall of 
the academic year in question and they plan to complete four years at Rice.
\end{compactenum}
\section{Acceptable Service Point Opportunities}
\begin{compactenum}
\item In order for service points to be awarded the opportunity must have been made 
equally available to all freshmen.
\item No committee may preference to its own members or any group when offering 
points.
\item Freshmen service points may not be awarded for any opportunity that involved 
any selection process other than �first come first served.�
\item Service points must directly benefit Duncan College.  Service to the greater Rice, 
Houston, American, or worldwide communities is encouraged but will not be 
counted for freshmen service points.
\item All service points must be announced to the college, approved by the VP in 
charge of the committee or representative, and approved by the LVP at least a 
month before the date of the opportunity.
\item The college must offer a variety of service point opportunities.  These 
opportunities must include some that are not related to alcohol and some that 
are offered during the daytime.  Failure to seize these opportunities will not be 
regarded as an excuse for failure to meet the quotas.
\end{compactenum}
\section{Establishing the Point Quotas}
\begin{compactenum}
\item There are two quotas for service points.  The first quota is called the On-Campus 
Quota, the second is called the Freshman Point Quota.
\item The On-Campus Quota will be approximately half of the Freshman Point Quota.
\item In collaboration with the VPs and committee heads, the LVP shall establish the 
point quotas for that year before the second week of classes of the academic 
year in question.
\item The quotas must be designed so that there is ample opportunity for all freshmen 
to fulfill the Freshman Point Quota
\end{compactenum}
\section{Registering for Service Opportunities}
\begin{compactenum}
\item The registration for all service opportunities must be made available to the 
college at least a week before the event.  This can be overridden in extreme 
cases by the Legislative Vice President 
\item In the event that a student registers for a service opportunity and then fails to 
satisfactorily complete that service (at the discretion of the party offering the opportunity) they will lose the same number of points they otherwise would have 
gained.  It is therefore possible to have negative points.
\item If a student needs to cancel their registration for a service point opportunity they 
must do so at least 72 hours before their assigned time.
\end{compactenum}
\section{Failure to Meet the On-Campus Quota}
\begin{compactenum}
\item In the event that a student fails to meet the On-Campus Quota they will not 
receive on-campus housing their Sophomore year.
\item Failure to meet the On-Campus Quota will not negate any other bump-exempt 
status.
\end{compactenum}
\section{Failure to Meet the Freshman Point Quota}
\begin{compactenum}
\item A student may never go into room draw with more than zero points if they have 
not completed their freshman service points.
\item If a student fails to meet their Freshman Point Quota during their freshmen year, 
they may complete those points in any subsequent year to regain their seniority 
housing status.  
\item As soon as a student fulfills their Freshman Point Quota they will go into all future 
rounds of room draw with the number of points listed below.
\begin{compactitem}
\item Anyone below the Freshman Point Quota = 0
\item Rising Sophomore = 2
\item Rising Junior = 1
\item Rising Senior =0
\end{compactitem}
\end{compactenum}

\end{document}  