\documentclass[11pt]{amsart}
\usepackage[margin=.5in]{geometry}                % See geometry.pdf to learn the layout options. There are lots.
\geometry{letterpaper}                   % ... or a4paper or a5paper or ... 
%\geometry{landscape}                % Activate for for rotated page geometry
%\usepackage[parfill]{parskip}    % Activate to begin paragraphs with an empty line rather than an indent
\usepackage{graphicx}
\usepackage{amssymb}
\usepackage{epstopdf}
\usepackage{paralist}
\DeclareGraphicsRule{.tif}{png}{.png}{`convert #1 `dirname #1`/`basename #1 .tif`.png}

\title{Constitution\_Court}
\author{Duncan College}
%\date{}                                           % Activate to display a given date or no date

\begin{document}
\maketitle
%\section{}
%\subsection{}

\begin{enumerate}
%%%%%%%%%%%%%%%%%%%%%%%%%%%%%%%%%%%%%%%%%%%%%%%%%%%%%
\item Organization
	\begin{enumerate}
		\item The Court shall consist of nine Justices. One of these justices shall be the Chief Justice. The other Justices shall be called Associate Justices. The Legislative Vice- President shall act as an advocate and be associated with the Court but not be a member.
			\item Requirements of Office
			\begin{enumerate}
			\item All Justices shall be residents of Duncan College
			\item The Justices shall not be on any type of probation.
			\item The Justices shall not hold any voting Forum office (save for the Chief Justice position in Forum) or be the President of Duncan College.
			\item Requirements of office, except Subsection 2 above, must be met only during the term of office.
			\end{enumerate}
		\item Selection of the Court and Court Officiers
			\begin{enumerate}
			\item The Chief Justice shall be elected by the entire College at the same time and in the same manner as the President and Legislative Vice-President and can only serve one position on the Court.
			\item The candidates for Chief Justice are encouraged to become acquainted with everyone in the College.
			\item All candidates for the position of Associate Justice, shall submit their applications to the Chief Justice following the election of the new Forum and prior to its second meeting.
			\item The selection of court members shall be made by a process of application to the Chief Justice. Individuals may submit applications to the Chief Justice, who shall fill positions on the Court based on qualification. The Chief Justice shall select 8 Associate Justices and should give consideration to, though not be limited by, a class distribution along the following guidelines: two justices from the rising senior class; two from the rising junior class; and two from the rising sophomore class. In addition, the Chief Justice should consider selecting at least one incoming student for the remaining two vacancies. The Chief Justice shall submit the names of the recommended Associate Justices to the Master(s) for approval by the beginning of the last week of classes for the spring semester. If any of the recommendations are rejected by the Master(s), the Chief Justice shall select different individuals for the vacant positions. They shall report the new recommendations to the Master(s) as soon as possible to be approved in the same manner.
			\item  Filling Vacancies on the Court:
			\begin{enumerate}
				\item If a vacancy occurs on the court in the position of Associate Justice, a process similar to that specified in Subsection 5 above will be used to select a replacement.
				\item If a vacancy occurs in the office of Chief Justice, a new Chief Justice shall be elected in a special election under the procedures outlined above in Subsections 1,2, and 3
			\end{enumerate}
			\end{enumerate}
		\item Removal of Associate Justices and Chief Justice
			\begin{enumerate}
			\item Impeachment of Associate Justices shall be initiated by an accusation submitted to the President of the College by any member of Duncan College.
			\item Upon a 4/5 vote of the entire Forum in favor of removal, the impeached Justice shall be removed and his his/her position shall be immediately declared vacant. If the impeachment petition has been signed by at least five Justices, then a simple majority of the Forum shall suffice to remove.
			\item Impeachment of the Chief Justice shall be initiated by an accusation submitted to the President of the College by any member of Forum or any of the Associate Justices. Upon a 4/5 vote of the entire Forum in favor of the removal, the impeached Chief Justice shall be removed and his/her position shall be immediately declared vacant.
			\item When information from a closed trial is relevant to the guilt or innocence of the Chief Justice, that information will be disclosed only to the voting members of the Forum and President at the discretion of the Master(s).
			\item Impeachment of the Legislative Vice-President shall be according to the regulations set in the Duncan College Constitution.
			\end{enumerate}
		
	\end{enumerate}
	
%%%%%%%%%%%%%%%%%%%%%%%%%%%%%%%%%%%%%%%%%%%%%%%%%%%%%
\item Jurisdiction
	\begin{enumerate}
		\item The Court shall have jurisdiction over resident and non-resident members of Duncan College involved in infractions of Duncan College rules and/or Rice University rules, or involved in conduct contrary to the standards necessary for responsible college life.	
		\item  In matters of all-school importance, the Court may, at its discretion refer cases to the University Court by a majority vote.
	\end{enumerate}
%%%%%%%%%%%%%%%%%%%%%%%%%%%%%%%%%%%%%%%%%%%%%%%%%%%%%
\item Powers and Duties
	\begin{enumerate}
	\item The Court has the power to investigate, to hold hearings, to hold trials, and to determine decisions and/or sentences.
	\item The Legislative Vice-President may serve as counsel to the defendant if the defendant so chooses. The primary duties of the Legislative Vice-President shall be 
	\begin{enumerate}
		\item to orient the defendant as to the trial process. 
		\item to explain to the defendant his his/her rights during the trial, and 
		\item during the trial, to point out facts in the defendant's favor which the Justices do not seem to understand or be aware of, but which the Legislative Vice-President deems important to the interest of the defendant.
	\end{enumerate}
	\item The Court has the duty to investigate grievances and gather relevant evidence for trials.
	\item The Chief Justice has the duty to maintain relations with RUPD, specifically the Duncan College officer.

	\end{enumerate}
%%%%%%%%%%%%%%%%%%%%%%%%%%%%%%%%%%%%%%%%%%%%%%%%%%%%%
\item Operations
	\begin{enumerate}
		\item Initiation of Proceedings
			\begin{enumerate}
			\item Infractions may be reported to any Justice by any member of Duncan College, by the College Master(s), or by the Assistant Dean for Student Judicial Programs. Infractions reported by persons other than those listed above shall be turned in to the College Master(s) or the College President, who shall then file a complaint in the name of the person entering the complaint. Complaints may be filed in the name of the College by any Forum officer.
			\item Formal complaints may not be withdrawn.
			\item Copies of all complaints and accusations shall be delivered to the College Master(s) or his/her/their designated representative before any trial proceedings are conducted.
			\item The College Master(s) shall be notified in advance of each Court action to ascertain whether information from his/her files is pertinent.
			\end{enumerate}
		\item Trial 
			\begin{enumerate}
			\item The Legislative Vice-President should inform the accused student at least seven days before the trial of the impending trial and the details surrounding the accusation and the trial. The Legislative Vice-President shall discuss the charges against the accused student, the evidence that has been collected at that point, the time and location of the trial, and the rights of the accused. If the accused student cannot attend the set trial time, the Court shall work with the Legislative Vice-President to set a time that works both for the student and for members of the Court. As the Court collects evidence leading up to the day of the trial, the Legislative Vice-President shall discuss new evidence with the accused student before the trial
			\item A trial shall be held within ten days, of notifying the accused student of the charges, excluding University holidays and exam periods, unless for valid reasons postponement is agreed upon by the accuser, the accused, and a majority of the Court.
			\item The accused shall have his/her rights explained to him/her in detail by the Legislative Vice-President. Any questions regarding the trial, the rights, or the accusation may be addressed through the Legislative Vice-President to the Chief Justice.
			\item The trial shall be constructed so that the facts of the case may be efficiently obtained and a just decision reached, but at no time violate the rights of the accused nor the integrity of the Court.
			\item The Court may strike from the records testimony it deems irrelevant.
			\item Names of those involved in a closed trial must remain secret at all times. 7. At least five Justices must be present to open a trial. At least five of the Justices originally present must be present for the entire trial
			\item With the consent of all the other Justices present, members of the Court may abstain from participation in the trial. This shall be done only in the interest of impartiality. If, due to this process or any other legitimate
reasons, less than five Justices remain eligible to attend the trial, then a sufficient number of the voting members of the Forum shall serve on the Court so as to maintain a five-member Court. The order in which the members of the Forum shall be asked to serve shall be: Internal Vice- President, External Vice-President, and an alternating selection between Treasurers and Secretaries, beginning with Treasurers.
			\item If the accused is a Justice of the Court, then he must abstain from the trial. If the Chief Justice abstains, then the remaining Justices shall select one of their number to preside.
			\item The accused may enter a plea of "guilty" or "not guilty" prior to the trial. Failure to enter a plea will be entered as a plea of "not guilty".
			\item Witnesses may be called or recalled by the accused or by the Court.
			\item The accuser must submit a deposition for the Court to read.
			\item A separate verdict must be reached for each accused person.
			\item If, after proper notification, the accused fails to attend the trial, the proceedings may be carried out in his/her absence. He/she forfeits all rights and may be tried accordingly.
			\end{enumerate}
		\item Verdict
			\begin{enumerate}
			\item The verdict shall be determined immediately after the trial in a closed meeting of the Court.
			\item A unanimous vote of the Justices present during the entire trial is necessary for a verdict of "guilty".
			\item If a verdict of "guilty" is not reached by the Court, the accused is "not guilty".
			\end{enumerate}
		\item Sentence
			\begin{enumerate}
			\item Upon a plea or verdict of "guilty", the Court will immediately determine the sentence of the accused.
			\item A 4/5 vote of the Justices present for the entire proceedings is necessary to determine a sentence or to reprimand the accused.
			\item After a sentence has been determined and before it is implemented, the accused and the College Master(s) shall be given a written notification of the action by the Court.
			\item A sentence may be enforced only after adequate time for appeal has elapsed.
			\end{enumerate}
		\item Records
			\begin{enumerate}
			\item Records of the entire proceedings shall be kept by the Chief Justice in a confidential file, open only to the Justices, the Master(s), the College President, and the University Proctor.
			\item Abstracts of hearings and trial shall be kept by the Chief Justice and made available to the College members upon request. Abstracts shall be posted publicly before the execution of sentences and shall remain posted for the period of one week.
			\end{enumerate}
		\item Appeals
			\begin{enumerate}
			\item The defendant or College Master(s) may appeal the decision if he/she
believes the verdict or sentence was reached in an unfair manner or if he/she believes the penalty is too severe. The appeal must be submitted through the Legislative Vice-President to the Chief Justice within five days after the trial, not including University holidays or exam days.
		\item The committee for an appeal of procedural error or unfair penalty shall consist of the Chief Justice, the College Master(s), and the College President. They shall review all relevant evidence, an abstract of the Court trial, and may hear additional testimony from the accused student. The committee can overturn the ruling of the Court or alter the penalty by a 2/3 vote. The President and Chief Justice shall each have one vote and the master or masters shall have a total of one vote. The decision of the appeals committee is final.
		\item The defendant or College Master(s) may appeal the decision of the court if he/she believes that there were relevant extenuating circumstances that he/she did not reveal during the trial for reasons of privacy. Such an appeal will be heard by the college Master(s) who may overturn the ruling of the court if he/she/they believe there is cause to do so.
			\end{enumerate}

	\end{enumerate}
%%%%%%%%%%%%%%%%%%%%%%%%%%%%%%%%%%%%%%%%%%%%%%%%%%%%%
\item Rights of the Accused
	\begin{enumerate}
	\item The accused may dispute or review any testimony or evidence given.
	\item The accused has the right to be present, if he/she desires, when all evidence and testimony from witnesses he/she requests are presented.
	\item The accused has the right to sum up the case before the Court decides the verdict.
	\item The accused may call or recall witnesses; however, no character witnesses may be called.
	\item The accused has the right to utilize the Legislative Vice-President as an advocate.
	\item The accused has the right to be in contact with the Legislative Vice-President and have him/her present at all times during the trial and preceding meetings.
	\item The accused has the right to request the Court to strike testimony from the records if he/she deems it irrelevant.
	\item The accused has the right of open trial if he/she so desires.
	\item The accused has the right to appeal any verdict or sentence rendered by the Court.
	\end{enumerate}
	
	\end{enumerate}
\end{document}  